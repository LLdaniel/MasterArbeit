\documentclass[pdftex, a4paper, parskip=full*, open=any, BCOR=10mm, fontsize=12pt, DIV=12, headsepline, footsepline=true, footinclude=false, draft=false, captions=nooneline]{scrbook}  %pdftex,
%
%\usepackage[utf8]{inputenc}
\usepackage[T1]{fontenc}
\usepackage{lmodern}
\usepackage{amsmath, amsthm, amssymb}
\usepackage{mathtools}
\usepackage{graphicx}
\usepackage{array}
\usepackage{babelbib}
\usepackage{verbatim}
\usepackage{lscape}
\usepackage{textcomp}    %fuer aufrechte mu
\usepackage[english]{babel}
\usepackage{caption}%Ist Voraussetzung für das Subcaption package, welches die subfigures erlaubt
\usepackage{subcaption}%Für Einbindung von Bildern als Untergraphiken subfig(ure) sind veraltet!
\renewcommand*{\chapterformat}{%Chapter Design
  \enskip\mbox{\scalebox{2}{\thechapter\autodot}}}
\renewcommand\chapterlinesformat[3]{%
  \parbox[b]{\textwidth}{\textcolor{royalazure}{\hrulefill#2}}\par%
  #3\par\bigskip%
  \textcolor{royalazure}{\hrule}}
\RedeclareSectionCommand[afterskip=1.5\baselineskip]{chapter}
%\usepackage[numbers]{natbib}
\usepackage{hyperref}
\usepackage{url}
%\usepackage{ziffer} %für Kommas bei Zahlen
\usepackage{multirow}
\usepackage[table]{xcolor}
%\usepackage{bibgerm}
%Aus dem Praktikum...
\usepackage{ae}                %macht schöneres ß
\usepackage[margin=10pt,font=small,labelfont=bf]{caption} %macht die Bildbeschriftungen richtig
\usepackage{epsfig}%Zur Einbindung von .eps: eps-->pdf
\usepackage{rotating}%für Querformat einer Tabelle 
\usepackage[binary-units=true]{siunitx} %SI unit package mit binary units = bits, bytes
%\renewcommand{\figurename}{Abb.}
\setlength\parindent{1em}%Rückt am Absatzbeginn ein
%......ENDE
\renewcommand{\thefootnote}{\fnsymbol{footnote}}%Symbole für Fußnoten und keine default arabics
\makeatletter%Clear counter nach jedem Kapitel für die Fußnoten
\@addtoreset{footnote}{section}% ""
\makeatother% ""
\addtokomafont{caption}{\small}
\setkomafont{captionlabel}{\bfseries \sffamily}
%\bibliographystyle{gerplain} %Ist jetzt im Hauptdokument verbaut-->s. \bibliography{}
\definecolor{royalazure}{rgb}{0.0, 0.22, 0.66}
\renewcommand*\chapterheadstartvskip{\vspace*{-\topskip}}%Seitenrandanfang einstellen
\renewcommand*\chapterheadendvskip{%Seitenrandende einstellen
  \vspace*{1\baselineskip plus .1\baselineskip minus .167\baselineskip}}

\newcommand{\ATLAS}{\textsc{atlas}}%for small capitals, looks even nicer
\newcommand{\CERN}{\textsc{cern}}
\newcommand{\LHC}{\textsc{lhc}}
\newcommand{\ALICE}{\textsc{alice}}
\newcommand{\CMS}{\textsc{cms}}
\newcommand{\LINAC}{\textsc{linac}}
\newcommand{\SUSY}{\textsc{susy}}
\newcommand{\GUT}{\textsc{gut}}
\newcommand{\GEANT}{\textsc{geant}}
\newcommand{\ROOT}{\textsc{root}}
\newcommand{\aMCNLO}{M\textsc{adgraph5}\textunderscore aMC\textsc{nlo}}
\newcommand{\POWHEG}{P\textsc{owheg}-B\textsc{ox}}
\newcommand{\EvtGen}{E\textsc{vt}G\textsc{en}}
\newcommand{\Pythia}{P\textsc{ythia}8.2}
\newcommand{\NNPDFd}{\textsc{NNPDF}3.0\textsc{NLO}}
\newcommand{\NNPDFz}{\textsc{NNPDF}2.3\textsc{NLO}}
\hyphenation{
  %Aus-gangs-sig-nal Be-trach-tet er-folg-reich Über-gangs-strah-lungs-de-tek-tor
  po-si-tiv
}

% Keine "Schusterjungen"
	\clubpenalty = 9999
% Keine "Hurenkinder"
	\widowpenalty = 9999 \displaywidowpenalty = 9999
