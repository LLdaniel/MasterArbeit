%%      --- TO DO --- 
%%    Die Frage ist, ob das jetzt noch eine Idee ist, den Detektoraufbau von der systematischen Teilchenbestimmungs-Seite aufzuziehen, quasi so, wie man es in der Masterclass lernt, mit einzelnen Spuren unterscheiden usw... --> siehe hierfür das Skript zusammenfassung ATLAS sowie https://cds.cern.ch/record/1096081/export/hx?ln=en und https://cds.cern.ch/record/1096081, oder ob das nicht dann doch zu viel wird
%%
%%    Danksagung --> noch überlegen, ob die nicht vielleicht doch auf Deutsch sein soll...
%-----------------------------------------------------------------------------------------------
%                   Gliederungsideen
%Experiment-Part: LHC, ATLAS, LQpairprod @pp collisions
%
%Software-Part: Datenprozessierung, MC Simulation, b-tagging, anti-kT(jet reconstruction), tau reconstruction, e+µ detektion?
%
%Theorie-Part: Standardmodell, beyond SM?(eigentlich klein hakten und eher versteckt von hinten rum in Theorie LQ einführen...), Theorie LQ(begründung, was es erkärt, so bisschen die beyond SM Schiene), 
%
%Ausgangspunkt, Forschungsfrage...
%
%
%
\documentclass[pdftex, a4paper, parskip=full*, open=any, BCOR=10mm, fontsize=12pt, DIV=12, headsepline, footsepline=true, footinclude=false, draft=false, captions=nooneline]{scrbook}  %pdftex,
%
%\usepackage[utf8]{inputenc}
\usepackage[T1]{fontenc}
\usepackage{lmodern}
\usepackage{amsmath, amsthm, amssymb}
\usepackage{mathtools}
\usepackage{graphicx}
\usepackage{array}
\usepackage{babelbib}
\usepackage{verbatim}
\usepackage{lscape}
\usepackage{textcomp}    %fuer aufrechte mu
\usepackage[english]{babel}
\usepackage{caption}%Ist Voraussetzung für das Subcaption package, welches die subfigures erlaubt
\usepackage{subcaption}%Für Einbindung von Bildern als Untergraphiken subfig(ure) sind veraltet!
\renewcommand*{\chapterformat}{%Chapter Design
  \enskip\mbox{\scalebox{2}{\thechapter\autodot}}}
\renewcommand\chapterlinesformat[3]{%
  \parbox[b]{\textwidth}{\textcolor{royalazure}{\hrulefill#2}}\par%
  #3\par\bigskip%
  \textcolor{royalazure}{\hrule}}
\RedeclareSectionCommand[afterskip=1.5\baselineskip]{chapter}
%\usepackage[numbers]{natbib}
\usepackage{hyperref}
\usepackage{url}
\usepackage{cancel} %for E/T miss durchgestrichen
%\usepackage{ziffer} %für Kommas bei Zahlen
\usepackage{multirow}
\usepackage[table]{xcolor}
%\usepackage{bibgerm}
%Aus dem Praktikum...
\usepackage{ae}                %macht schöneres ß
\usepackage[margin=10pt,font=small,labelfont=bf]{caption} %macht die Bildbeschriftungen richtig
\usepackage{epsfig}%Zur Einbindung von .eps: eps-->pdf
\usepackage{rotating}%für Querformat einer Tabelle 
\usepackage[separate-uncertainty=true, binary-units=true]{siunitx} %SI unit package mit binary units = bits, bytes
%\renewcommand{\figurename}{Abb.}
\setlength\parindent{1em}%Rückt am Absatzbeginn ein
%......ENDE
\renewcommand{\thefootnote}{\fnsymbol{footnote}}%Symbole für Fußnoten und keine default arabics
\makeatletter%Clear counter nach jedem Kapitel für die Fußnoten
\@addtoreset{footnote}{section}% ""
\makeatother% ""
\addtokomafont{caption}{\small}
\setkomafont{captionlabel}{\bfseries \sffamily}
%\bibliographystyle{gerplain} %Ist jetzt im Hauptdokument verbaut-->s. \bibliography{}
\definecolor{royalazure}{rgb}{0.0, 0.22, 0.66}
\renewcommand*\chapterheadstartvskip{\vspace*{-\topskip}}%Seitenrandanfang einstellen
\renewcommand*\chapterheadendvskip{%Seitenrandende einstellen
  \vspace*{1\baselineskip plus .1\baselineskip minus .167\baselineskip}}

\newcommand{\ATLAS}{\textsc{atlas}}%for small capitals, looks even nicer
\newcommand{\CERN}{\textsc{cern}}
\newcommand{\LHC}{\textsc{lhc}}
\newcommand{\ALICE}{\textsc{alice}}
\newcommand{\CMS}{\textsc{cms}}
\newcommand{\LINAC}{\textsc{linac}}
\newcommand{\SUSY}{\textsc{susy}}
\newcommand{\GUT}{\textsc{gut}}
\newcommand{\GEANT}{\textsc{geant}}
\newcommand{\ROOT}{\textsc{root}}
\newcommand{\aMCNLO}{M\textsc{adgraph5}\textunderscore aMC\textsc{nlo}}
\newcommand{\POWHEG}{P\textsc{owheg}-B\textsc{ox}}
\newcommand{\EvtGen}{E\textsc{vt}G\textsc{en}}
\newcommand{\Pythia}{P\textsc{ythia}8.2}
\newcommand{\NNPDFd}{\textsc{NNPDF}3.0\textsc{NLO}}
\newcommand{\NNPDFz}{\textsc{NNPDF}2.3\textsc{NLO}}
\hyphenation{
  %Aus-gangs-sig-nal Be-trach-tet er-folg-reich Über-gangs-strah-lungs-de-tek-tor
  po-si-tiv
}

% Keine "Schusterjungen"
	\clubpenalty = 9999
% Keine "Hurenkinder"
	\widowpenalty = 9999 \displaywidowpenalty = 9999

\begin{document}
\begin{titlepage}
  \vspace*{-7\baselineskip}
	\enlargethispage{100mm}
		\begin{center}
		\LARGE{\textbf{Master Thesis}\\}
		\vspace{3mm}	
		\textcolor{royalazure}{\noindent\rule{\textwidth}{3pt}}
		\huge{\textbf{Signal and background studies for scalar leptoquark pair production in the t$\bar{\textbf{t}}\,\mathbf{+\,2\tau}$ channel at the ATLAS experiment}\\}
		\vspace{3mm}
		\Large{Daniel Adlkofer\\}
		\textcolor{royalazure}{\noindent\rule{\textwidth}{3pt}}
		\vspace{3mm}
        \includegraphics[width=0.55\textwidth]{figures/neuSIEGEL.eps} \\
		\vspace{3mm}
		Supervisor \\
		\Large{Prof. Dr. Raimund Str\"{o}hmer\\}
               	\vspace{3mm}
                Advisor \\
    \Large{Dr. Mahsana Haleem\\}
               	\vspace{3mm}

		\vspace{5mm}
		December 2018\\
		  \noindent\hrulefill\\
		\vspace{3mm}
 		Lehrstuhl f\"{u}r Physik und ihre Didaktik\\
 		Physikalisches Institut\\
	    Julius-Maximilians-Universit\"{a}t W\"{u}rzburg
	\end{center}
\end{titlepage}
\cleardoubleoddemptypage

\tableofcontents
%%%%%%%%%%%%%%%%%%%%%%%%%%%%%%%%%%%%%%%%%%%%%%%%%%%%%%%%%%%%%%%%%%%%%%%%%%%%%%%%%%%%%%%%%%%%%%%%%%%
\chapter{XyZ}
%%%%%%%%%%%%%%%%%%%%%%%%%%%%%%%%%%%%%%%%%%%%%%%%%%%%%%%%%%%%%%%%%%%%%%%%%%%%%%%%%%%%%%%%%%%%%%%%%%%
\begin{table}[htbp]
		\centering
		\begin{tabular*}{\linewidth}{@{\extracolsep{\fill}}ccccc}
		\hline
		\hline
		\rule[-6pt]{0pt}{21pt} \textbf{sample}  & \textbf{t$\bar{\textbf{t}}$}  & \textbf{t$\bar{\textbf{t}}$H} & \textbf{LQ$_{\SI{500}{\giga\electronvolt}}$} & \textbf{LQ$_{\SI{1}{\tera\electronvolt}}$}
		\\
		\hline
		\rule[-7pt]{0pt}{23pt} selection  & reconstruction & reconstruction & reconstruction & reconstruction  
		\\ 
		\rule[-7pt]{0pt}{23pt}  & event yield & event yield & event yield & event yield 
		\\
		\hline
		\rule[-6pt]{0pt}{21pt} $\geq 2\,$b-jets   & $186\,395$ & $209$ & $152$ & $1.5$
		\\
		\rule[-6pt]{0pt}{21pt} $\geq 2\,$b-jets $+\geq1\,\tau$  & $505$ & $7$ & $94$ & $0.9$
		\\
		\rule[-6pt]{0pt}{21pt} $\geq 2\,$b-jets $+\geq2\,\tau$ & $1.7$ & $0.4$ & $27$ & $0.2$ 
		\\
		\hline
		\hline
		\end{tabular*}
		\caption[Event yield for the t$\bar{\text{t}}$, t$\bar{\text{t}}$H and the LQ samples.]{Event yield for different selections with tau leptons for the t$\bar{\text{t}}$, the t$\bar{\text{t}}$H and the LQ Monte Carlo sample. The luminosity account for $150\,\text{fb}^{-1}$.}
		\label{ttHttbarEvent}
	\end{table}
%
\begin{table}[htbp]
		\centering
		\begin{tabular*}{\linewidth}{@{\extracolsep{\fill}}ccc}
		\hline
		\hline
		\rule[-6pt]{0pt}{21pt} \textbf{sample}  & \textbf{t$\bar{\textbf{t}}$} & \textbf{t$\bar{\textbf{t}}$H}
		\\
		\hline
		\rule[-7pt]{0pt}{23pt} selection  & efficiency $\frac{\epsilon}{\%}$ & efficiency $\frac{\epsilon}{\%}$ 
		\\
		\hline
		\rule[-6pt]{0pt}{21pt} $\geq 2\,$b-jets & $26.52$ & $36.72$ 
		\\
		\rule[-6pt]{0pt}{21pt} $\geq 2\,$b-jets $+1\,\tau$  & $3.18$ & $8.83$ 
		\\
		\rule[-6pt]{0pt}{21pt} $\geq 2\,$b-jets $+2\,\tau$  & $1.41$ & $2.13$ 
		\\
		\hline
		\hline
		\end{tabular*}
		\caption[Efficiencies for the t$\bar{\text{t}}$ and the t$\bar{\text{t}}$H sample.]{Efficiencies for different selections with tau leptons for the t$\bar{\text{t}}$ and the t$\bar{\text{t}}$H Monte Carlo sample.}
		\label{ttHttbarEff}
	\end{table}
%
%
%
%
\begin{table}[htbp]
		\centering
		\begin{tabular*}{\linewidth}{@{\extracolsep{\fill}}cccccc}
		\hline
		\hline
		\rule[-6pt]{0pt}{21pt} \textbf{sample} & & \multicolumn{2}{c}{\textbf{t$\bar{\textbf{t}}$}}  & \multicolumn{2}{c}{\textbf{t$\bar{\textbf{t}}$H}} 
		\\
		\hline
		\rule[-7pt]{0pt}{23pt} \multirow{2}{*}{selection} & reference & reconstruction & truth & reconstruction & truth  
		\\ 
		\rule[-7pt]{0pt}{23pt} & selection & ratio $\frac{r}{\%}$ & ratio $\frac{r}{\%}$ & ratio $\frac{r}{\%}$ & ratio $\frac{r}{\%}$ 
		\\
		\hline
		\rule[-6pt]{0pt}{21pt} $\geq 2\,$b-jets $+1\,\tau$ & $\geq 2\,$b-jets & $0.28$ & $2.35$ & $3.43$ & $14.26$
		\\
		\rule[-6pt]{0pt}{21pt} $\geq 2\,$b-jets $+2\,\tau$ & $\geq 2\,$b-jets & $0.0011$ & $0.020$ & $0.24$ & $4.11$ 
		\\
		\hline
		\hline
		\end{tabular*}
		\caption[Ratios for the t$\bar{\text{t}}$ and the t$\bar{\text{t}}$H sample.]{Ratios for different selections with tau leptons for the t$\bar{\text{t}}$ and the t$\bar{\text{t}}$H Monte Carlo sample.}
		\label{ttHttbarRatio}
	\end{table}
%	
%
%	
%	
	\begin{table}[htbp]
		\centering
		\begin{tabular*}{\linewidth}{@{\extracolsep{\fill}}ccccc}
		\hline
		\hline
		\rule[-6pt]{0pt}{21pt} \textbf{sample}  & \multicolumn{2}{c}{\textbf{t$\bar{\textbf{t}}$}}  & \multicolumn{2}{c}{\textbf{t$\bar{\textbf{t}}$H}} 
		\\
		\hline
		\rule[-7pt]{0pt}{23pt} \multirow{2}{*}{selection}  & numerator      & denominator & numerator      & denominator
		\\ 
		\rule[-7pt]{0pt}{23pt}                             & event yield    & event yield & event yield    & event yield 
		\\
		\hline
		\rule[-6pt]{0pt}{21pt} truth matching for tau      & $63$            & $13723$      & $5590$        & $21610$
		\\
		\rule[-6pt]{0pt}{21pt} efficiency                  & \multicolumn{2}{c}{$0.46\%$}    & \multicolumn{2}{c}{$25.9\%$}
		\\
		\hline
		\rule[-6pt]{0pt}{21pt} tau from H$^0$, W$^{\pm}$, Z$^0$& $0$        & $0$         & $4859$          & $11988$
		\\
		\rule[-6pt]{0pt}{21pt} efficiency                  & \multicolumn{2}{c}{-}   & \multicolumn{2}{c}{$40.5\%$}
		\\
		\hline
		\rule[-6pt]{0pt}{21pt} tau from B-mesons           & $63$            & $13722$      & $20$            & $7416$ 
		\\
		\rule[-6pt]{0pt}{21pt} efficiency                  & \multicolumn{2}{c}{$0.46\%$}   & \multicolumn{2}{c}{$0.27\%$}
		\\
		\hline
		\rule[-6pt]{0pt}{21pt} tau within a jet            & $8440$         & $3776952$      & $18511$         & $20327225$ 
		\\
		\rule[-6pt]{0pt}{21pt} efficiency                  & \multicolumn{2}{c}{$0.22\%$}   & \multicolumn{2}{c}{$0.091\%$}
		\\
		\hline
		\rule[-6pt]{0pt}{21pt} tau within a b-jet          & $6098$        & $2658379$      & $2317$         & $1208924$ 
		\\
		\rule[-6pt]{0pt}{21pt} efficiency                  & \multicolumn{2}{c}{$0.23\%$}   & \multicolumn{2}{c}{$0.19\%$}
		\\
		\hline
		\hline
		\end{tabular*}
		\caption[Event yield for the t$\bar{\text{t}}$ and the t$\bar{\text{t}}$H sample.]{Event yield for different selections with tau leptons for the t$\bar{\text{t}}$ and the t$\bar{\text{t}}$H Monte Carlo sample. The luminosity account for $36.1\,\text{fb}^{-1}$.}
		\label{ttHttbarEventTruthMatching}
	\end{table}
%
%
%
%
	\begin{table}[htbp]
		\centering
		\begin{tabular*}{\linewidth}{@{\extracolsep{\fill}}ccccc}
		\hline
		\hline
		\rule[-6pt]{0pt}{21pt} \textbf{sample}  & \multicolumn{2}{c}{\textbf{LQ${_{\mathbf{\SI{500}{\giga\electronvolt}}}}$}}  & \multicolumn{2}{c}{\textbf{LQ${_{\SI{1}{\tera\electronvolt}}}$}} 
		\\
		\hline
		\rule[-7pt]{0pt}{23pt} \multirow{2}{*}{selection}  & numerator      & denominator & numerator      & denominator
		\\ 
		\rule[-7pt]{0pt}{23pt}                             & event yield    & event yield & event yield    & event yield 
		\\
		\hline
		\rule[-6pt]{0pt}{21pt} truth matching for tau      & $2604$            & $5362$      & $2263$        & $5055$
		\\
		\rule[-6pt]{0pt}{21pt} efficiency                  & \multicolumn{2}{c}{$48.6\%$}    & \multicolumn{2}{c}{$44.8\%$}
		\\
		\hline
		\rule[-6pt]{0pt}{21pt} tau from H$^0$, W$^{\pm}$, Z$^0$& $95$        & $340$         & $82$          & $461$
		\\
		\rule[-6pt]{0pt}{21pt} efficiency                  & \multicolumn{2}{c}{$27.9\%$}   & \multicolumn{2}{c}{$17.8\%$}
		\\
		\hline
		\rule[-6pt]{0pt}{21pt} tau from B-mesons           & $0$            & $183$      & $0$            & $200$ 
		\\
		\rule[-6pt]{0pt}{21pt} efficiency                  & \multicolumn{2}{c}{$0.0\%$}   & \multicolumn{2}{c}{$0.0\%$}
		\\
		\hline
		\rule[-6pt]{0pt}{21pt} tau from LQ                 & $1744$            & $3286$      & $1057$            & $2022$ 
		\\
		\rule[-6pt]{0pt}{21pt} efficiency                  & \multicolumn{2}{c}{$53.1\%$}   & \multicolumn{2}{c}{$52.3\%$}
		\\
		\hline
		\rule[-6pt]{0pt}{21pt} tau within a jet            & $7232$         & $55208$      & $7011$         & $63671$ 
		\\
		\rule[-6pt]{0pt}{21pt} efficiency                  & \multicolumn{2}{c}{$13.1\%$}   & \multicolumn{2}{c}{$11.0\%$}
		\\
		\hline
		\rule[-6pt]{0pt}{21pt} tau within a b-jet          & $2317$        & $1208924$      & $6098$         & $2658379$ 
		\\
		\rule[-6pt]{0pt}{21pt} efficiency                  & \multicolumn{2}{c}{$0.45\%$}   & \multicolumn{2}{c}{$0.23\%$}
		\\
		\hline
		\hline
		\end{tabular*}
		\caption[]{}
		\label{LQEventTruthMatching}
	\end{table}
%%%%%%%%%%%%%%%%%%%%%%%%%%%%%%%%%%%%%%%%%%%%%%%%%%%%%%%%%%%%%%%%%%%%%%%%%%%%%%%%%%%%%%%%%%%%%%%%%%%%%%%%%%%%%%%%%%%%%%%%%%%
\chapter{Introduction}
%einen gescheiten Einleitungsgedanken überlegen! Ja nicht Higgs... :see_no_evil:
%Yukawa Kopplung vielleicht? --> ich bin mir nicht mehr sicher, aber vielleicht habe ich da einen Artikel gelesen (News CERN oder so) wo es um die noch bessere Messung der Yukawa Kopplung ging...
%%%%%%%%%%%%%%%%%%%%%%%%%%%%%%%%%%%%%%%%%%%%%%%%%%%%%%%%%%%%%%%%%%%%%%%%%%%%%%%%%%%%%%%%%%%%%%%%%%%%%%%%%%%%%%%%%%%%%%%%%%%%%
\chapter{Theoretical background for the search of scalar leptoquarks}\label{theory}
\section{The Standard Model of particle physics}\label{SM}
%Gedanken zum SM:
%von folgender Seite betrachten: SM vom Standpunkt der Symmetrien und Gruppentheorie einführen und von der Seite aufziehen.
%Die Frage ist, ob der Lagrangian vorkommen soll
%%%%%%%%%%%%%%%%%%%%%%%%%%%%%%%%%%%%%%%%%%%%%%%%%%%%%%%%%%%%%%%%%%%%%%%%%%%%%%%%%%%%%%%%%%%%%%%%%%%%%%%%%%%%%%%%%%%%%%%%%%%%%
\chapter{Experimental setup for the search of scalar leptoquarks}\label{experiment}
For the search of scalar leptoquarks the ATLAS detector at the Large Hadron Collider (LHC) is used as experimental setup which will be described within this chapter. In section \ref{LHC} the general setting of the proton-proton collider located at the CERN research center is the subject of interest. The particle detection of the resulting collision events will take place in the ATLAS detector with its different specialized components (section \ref{ATLAS}). Section \ref{LQpp} addresses the leptoquark pair production in proton-proton collisions.  
%%%%%%%%%%%%%%%%%%%%%%%%%%%%%%%%%%%%%%%%%%%%%%%%%%%%%%%%%%%%%%%%%%%%%%%%%%%%%%%%
\section{The Large Hadron Collider accelerator complex}\label{LHC}
The research center CERN (Conseil Européen pour la Recherche Nucléaire) was founded in $1954$ near Geneva, Switzerland to become a major European joint venture on elementary particle physics. In the mean time $22$ member states are participating in that large-scale project with the ambition to probe the essential constitutes of nature and the fundamental forces acting between them. \cite{CERNabout}\par
%
\begin{figure}[htbp]                                 
 \begin{center}                                       
  \includegraphics[width=0.55\linewidth]{figures/CERNKomplex.jpg} 
   \caption[Schematic of the CERN accelerator complex.]{Schematic of the CERN accelerator complex with its different stages and few experiments like ATLAS located at one crossing point for protons. \cite{CERNKomplex}}
  \label{complex}                                     
 \end{center}
\end{figure}
%
In the huge accelerator complex protons reach through different stages energies of $\SI{6.5}{\tera\electronvolt}$ and will be brought to collisions at defined interaction sites in time intervals of $\SI{25}{\nano\second}$. Particle detectors then register signatures of the resulting collsion events and the analysis of new created particles gives insight to the nature of elementary particle physics.\newline 
Figure \ref{complex} shows the different acceleration stages. Starting from the injection protons will gain as much energy as $\SI{50}{\mega\electronvolt}$ in the linear accelerator LINAC2 and will be further transferred to the Proton Synchrotron Booster ($\SI{1.4}{\giga\electronvolt}$), the Proton Synchrotron ($\SI{25}{\giga\electronvolt}$), the Super Proton Synchrotron ($\SI{450}{\giga\electronvolt}$) and finally to the LHC ring with its $\SI{26.7}{\kilo\meter}$ circumference. \cite{CERNabout}\newline
The LHC is designed as two-ring proton-proton collider. Conditions for a stable proton beam are diversely including high vacua of $\SI{E-10}{\milli\bar}$ to $\SI{E-11}{\milli\bar}$ and temperatures of $\SI{1.9}{\kelvin}$ for the superconducting NbTi-magnets of the accelerator. \cite{LHCJINST}   
%hier dann weiter ins Detail wie LHC JINST #######################
%Die Idee ist eigentlich eine genauere Beschreibung der Komponenten, allerdings ist JINST schon wieder so tief... So überblicksmäßiger wäre viel bessser, bei dem man noch im Blick behalten kann, um was es eigentlich geht...
\par
Different more experiments like ALICE\cite{ALICE}, LHCb\cite{LHCb} are located at CERN due to the variety of research questions. But the subject of interest in this work lies in the high luminosity experiment ATLAS specialized for proton-proton collisions like its counterpart CMS\cite{CMS}. Main tasks of ATLAS are more precise measurements of the SM (see chapter \ref{SM}), better understanding Quantum Chromo Dynamics (QCD) or search for supersymmetric models and new physics. With the LHC production of $109$ inelastic events per second up to $23$ simultaneously events at dominating high QCD cross sections reqiure a powerful detector that is capable of recognizing the characteristic signatures. These circumstances make up the demands for ATLAS including fast electronic elements, high detector granularity, handling high particles fluxes and reducing overlapping events at a large acceptance and coverage region. \cite{ATLASJINST}       
%AußerdeM: Hinweis auf high Lumi LHC als Art mini-Ausblick in diesem Kapitel https://arxiv.org/pdf/1705.08830.pdf, https://cds.cern.ch/record/1711887/files/ATL-COM-UPGRADE-2014-014.pdf ################
%%%%%%%%%%%%%%%%%%%%%%%%%%%%%%%%%%%%%%%%%%%%%%%%%%%%%%%%%%%%%%%%%%%%%%%%%%%%%%%%%%%%%%%%%%%%%%%%%%%%%
\section{The ATLAS detector at the LHC}\label{ATLAS}
One of the general purpose detector for proton-proton collisions is the ATLAS detector. This $\SI{25}{\meter}$ tall detector is located at one interaction point of the LHC where bunches, consisting of approximately $\SI{E11}{}$ protons, collide at a rate of $\SI{40}{\mega\hertz}$ \cite{ATLASJINST}. The number of particles encountered per time is given by \cite{Perkins}
\begin{align}
                        \dot{N}=\mathcal{L}\sigma
\end{align}
with the cross section $\sigma$ for the present event and the instant luminosity $\mathcal{L}$. Given a measure for the number of collisions per unit time the instant luminosity can be introduced and is often used as key parameter in collider physics \cite{LHCJINST}.
\begin{align}
                        \mathcal{L}=\frac{N_bn_bf_{\text{rev}}\gamma_r}{4\pi\epsilon_n\beta^*}F
\label{Lumi}
\end{align}
Where $N_b$ is the number of particles per bunch, $n_b$ the number of bunches per beam, $f_{\text{rev}}$ the rotational frequency, $\gamma_r$ the Lorentz factor, $\epsilon_n$ the normalized transverse beam emittance, $\beta^*$ the betatron function at the collision point and $F$ respects the geometric luminosity reduction factor due to the crossing angle at the collision point. The design luminosity for ATLAS was exceeded with $\mathcal{L}=\SI{2.05E34}{\per\square\centi\meter\per\second}$ for $2.05$ times on the $2^{\text{nd}}$ of November 2017 emphasizing the great success over the years \cite{designLumiExceeded}.\par 
%Erklärung für Lumi auch aus Perkins? ##########
The aspiration to be sensitive to the great variety of particles governed by the fundamental forces (see chapter \ref{SM}) influenced the detector design accordingly. The layered structure reflects the fact that
%Motivation eher sogar von der Seite, wieso so gebaut, wie er ist --> Teilchen detektieren (Verweis auf SM Kaptiel)-> Schichtaufbau-> etc... #######
The basic structure of ATLAS is shown in figure \ref{ATLASandCoordinate} with its different sub-detector systems together with the convention for the used coordinate system.  %Koordinatensystem
%
\begin{figure}
 \centering
  \begin{subfigure}[c]{0.7\textwidth}
   \includegraphics[width=\textwidth]{figures/ATLASDesign.png}
    \subcaption[The structure of the ATLAS detecor and its sub-systems.]{The layered structure of the ATLAS Detector at the LHC with its sub-systems Inner Detektor, Calorimeter, magnets and Muon Spectrometer \cite{ATLASJINST}.}
   \label{ATLASDesign}
  \end{subfigure}
  \begin{subfigure}[c]{0.7\textwidth}
   \includegraphics[width=1\textwidth]{figures/ATLASDesignCoordinate.png}
   \subcaption[Definition of the global ATLAS coordinate system.]{The global ATLAS coordinate system formulated in cylindric coordinates with the $z$-axis parallel to the beam line and the transverse plane defined through azimuthal angle $\phi$ and pseudorapidity $\eta$. Based on \cite{ATLASJINST}.}
   \label{Coordinate}
  \end{subfigure}
 \caption[Structure of the ATLAS detector and the used coordinate system.]{Structure of the ATLAS detector and the used coordinate system.}
 \label{ATLASandCoordinate}
\end{figure}
%
The nominal interaction point acts as origin of the coordinate system where the $z$-axis follows the beam line. Perpendicular to the $z$ axis lies the transverse $x$-$y$-plane usually described through the azimuthal angle $\phi$. The positive $x$-axis points towards the center of the LHC. The cylindric symmetry of the detector suggests a cylindric coordinate system with the angle $\theta$ starting from the beamline. \cite{ATLASJINST} Since the polar angle is not a Lorentz invariant quantity it is useful to describe the position in terms of rapidity \cite{LHCJINST} $w=\frac12\ln{\frac{E+p_zc}{E-p_zc}}$ in that highly relativistic regime. In the limit of large momenta i.e. $|\mathbf{p}|c\approx E$ the rapidity coincides with the pseudorapidity formulated as \cite{ChinaPseudorapidityBook}
\begin{align}
                        \eta=-\ln{\tan\frac{\theta}{2}}
\label{pseudorapidity}
\end{align}
This variable has only the polar angle as dependence and is therefore the adequate quantity in the context of collision experiments where usually the angle $\theta$ from the beamline is measured. \cite{ChinaPseudorapidityBook}\par      
%Aufbau Idee: auch so ein Walkthrough durch die einzelnen Komponenten wie bei der Bachelorarbeit
%einzelne Komponenten + Trigger(ref auf Datenauswertung, Software part im Text dann)
\textbf{The magnet configuration} includes a superconducting selenoid with a field strength of $\SI{2}{\tesla}$ sourrounding the inner detector as well as three large superconducting toroid magnets around the calorimeter. The barrel toroid magnet delivers a field strength of $\SI{0.5}{\tesla}$ and in the end-cap a field of $\SI{1}{\tesla}$ prevails. \cite{ATLASJINST}\newline%Bending particle path -->identification... #######
\textbf{The inner detector} is responsible for pattern recognition, momentum and vertex measurements and electrically charged particle identification which is achieved with a combination of semiconductor pixel and microstrip trackers (SCT). Additional straw tube tracking detectors are sensitive to transistion radiation (TRT) in the outer part that are responsible for high vertex and momentum resolution. The $R-\phi$ segmented pixel detectors are of size $50\times \SI{400}{\square\micro\meter}$ and the SCTs with its $8$ strip layers cover together a range of $|\eta|<2.5$. Typically $36$ hits per track is provided by the $\SI{4}{\milli\meter}$ straw tubes of the TRTs and cover the range $|\eta|\leq 2.0$. \cite{ATLASJINST}\newline %Unterscheidung teilchen anhand der Übergangsstrahlung ####################
Liquid argon electromagnetic sampling \textbf{calorimeter} with high granularity allow an excellent energy performance for electrons and photons. It has a total thickness of more than $22$ radiation lengths $X_0$ in the barrel region ($|\eta|<1.475$) and more than $24X_0$ in the end-cap region ($1.375<|\eta|<3.2$). For hadronic energy measurements a scintillator-tile calorimeter covering $|\eta|<1.7$ is in operation. It is a sampling calorimeter and uses steel as absorber material and scintillating tiles as active material in conjunction with wavelength shifting fibres. Further LAr technology is used for hadronic particles in the outer pseudorapidity range up to $|\eta|=3.2$. Here copper plates provide the absorber material. The forward calorimeters extend the coverage for hadronic and electromagnetic energy measurements to $|\eta|=4.9$ and are $10X_0$ deep. \cite{ATLASJINST}\newline%Energiemessung: provide god res for high energy jets###################
The \textbf{muon system} is suited in the outer layer of ATLAS and provides as independent system resolution for high energy muon tracks with three layered precision chambers. This is possible because of the air-cored toroid magnet system including one barrel and two end-cap magnets generating strong bending power in a large volume and delivering  a mostly perpendicular magnetic field regarding the muon trajectories. The bending power $\int{\vec{B}d\vec{l}}$ along the muontrack $d\vec{l}$ reaches $\SI{1.5}{\tesla\meter}$ to $\SI{5.5}{\tesla\meter}$ in the range $|\eta|<1.4$ (barrel) and up to $\SI{7.5}{\tesla\meter}$ (end-cap). The precision chambers are Monitored Drift Tubes (MDT) and in the larger pseudorapidity range Chathode Strip Chambers (CSC) which are multiwire proportional chambers. Due to the fact that the overall performance depends crucially on the alignment of the muon detectors with repect to each other and the Inner Detector MDTs are equipped with a optical monitoring system with $1200$ sensors. Resistive Plate Chambers (RPC) and Thin Gap Chambers (TGC) are the constitutes of the muon trigger system. \cite{ATLASJINST} \newline %Myonen kommen überall durch-->hier detektieren + Problem der Spurkrümmung für high pt--> eigenes system für die genauere myon-messung ############
The data recording rate is limited due to technology and resource limitations and has to be reduced from $\SI{40}{\mega\hertz}$ to $\SI{200}{\mega\hertz}$. This poses high demands on an efficient \textbf{trigger system} which is organised in three levels. Level $1$ uses only a subset of the total detector information making basic decisions to flag so called regions of interest i.e. coordinate regions. Searches include patterns for high transverse momenta of muon tracks, electrons and photons as well as jets or large missing energy balances. The output rate after this first selection accounts for $\SI{75}{\kilo\hertz}$. The high level trigger $2$ and $3$ are responsible for selecting the level $1$ triggerd regions at full granularity and precision. The level $3$ event filter is the final stage and achieves data reduction down to the final data-taking rate of $\SI{200}{\hertz}$ writing events of the size of approximately $\SI{1.3}{\mega\byte}$ to the disks. The event filter`s selection criteria are implemented using offline analysis procedures. \cite{ATLASJINST}%efficient triggering with sufficient background rejection ##############
%%%%%%%%%%%%%%%%%%%%%%%%%%%%%%%%%%%%%%%%%%%%%%%%%%%%%%%%%%%%%%%%%%%%%%%%%%%%%%%%%%%%%%%%%%%%%%%%%%%%%%%
\section{Leptoquark pair production in proton-proton collisions}\label{LQpp}
%%%%%%%%%%%%%%%%%%%%%%%%%%%%%%%%%%%%%%%%%%%%%%%%%%%%%%%%%%%%%%%%%%%%%%%%%%%%%%%%%%%%%%%%%%%%%%%%%%%%%%%%%%%%%%%%%%
\section{b-tagging at ALTAS}\label{btagging}
%Grundsatzidee:
%allgemeines Kapitel hierzu: https://arxiv.org/abs/1709.01290, https://arxiv.org/abs/1111.4190 und https://arxiv.org/abs/0809.4896
% dann ganz kurzer Ausblick auf https://arxiv.org/abs/1711.08811 (sind alle noch nicht in der bib)
%%%%%%%%%%%%%%%%%%%%%%%%%%%%%%%%%%%%%%%%%%%%%%%%%%%%%%%%%%%%%%%%%%%%%%%%%%%%%%%%%%%%%%%%%%%%%%%%%%%%%%%%%%%%%%%%%%
\listoffigures
\addcontentsline{toc}{chapter}{List of figures}%fügt das Bildverzeichnis zum Inhaltsverzeichnis
\listoftables
\addcontentsline{toc}{chapter}{List of tables}%fügt das Tabellenverzeichnis zum Inhaltsverzeichnis
%%%%%%%%%%%%%%%%%%%%%%%%%%%%%%%%%%%%%%%%%%%%%%%%%%%%%%%%%%%%%%%%%%%%%%%%%%%%%%%%%%%%%%%%%%%%%%%%%%%%%%%%%%%%%%%%%%
%\chapter*{Acknowledgement}
%An dieser Stelle möchte ich meinen herzlichen Dank an die Personen aussprechen, die für das Gelingen dieser Arbeit unentbehrlich waren:
%\begin{itemize}
% \item Professor Thomas Trefzger für die sehr freundliche Aufnahme an den Lehrstuhl Physik und ihre Didaktik
% \item Professor Raimund Ströhmer für die konstruktiven Anregungen und hilfreichen Diskussionen, welche die Arbeit sehr vorangebracht haben 
%  \item Dr. Mahsana Haleem für die Betreuung meiner Bachelor Arbeit, für die geduldigen Erklärungen zu jeder Zeit und die lehrreichen Diskussionen zum Thema meiner Arbeit und darüber hinaus
%  \item Verena Herget für die Einarbeitung in die Datenanalyse, für tausend Antworten zu Fragen aller Art, ob phyiskalischer, programmiertechnischer oder alltäglicher Natur, für bereitwilliges und kompetentes Erklären sogar per Fernanalyse, für das Korrekturlesen der Arbeit, für die äußerst nette Atmosphäre im Büro zu jeder Zeit und den Aktionen freizeittechnischer Natur. Ohne Dich wäre ich niemals soweit gekommen. Danke! 
%  \item Deb Sankar Bhattacharya
%  \item Thorben Swirski für das Korrekturlesen der Arbeit, fröhliches debuggen, viele Anregungen nebenbei und für ausführliche, fachliche Erläuterungen
%  \item Robin Boshuis, die als sehr nette Bürokollegen die Motivation hochgehalten haben, oder zur richtigen Zeit auf eine Pause bestanden haben. Man muss ja mal Lego bauen oder Wein trinken ;)
%  \item Susan Fried und Florian Treisch für die gemeinsamen Stunden, die ich zu jeder Zeit genossen habe
%  \item Denise Böhm NTW
%  \item Stephan Lück Salat, Serien, Photo
%  \item Frank Finkenberg Weinfeste
%  \item allen anderen Mitgliedern des Lehrstuhl, die stets eine Wohlfühlatmosphäre geschaffen haben und bei kleineren Problemen hilfsbereit zur Seite standen. Danke auch für die lustigen Runden beim Mittagessen =)
%  \item meiner engeren Familie allen voran meinen Eltern und Geschwistern, die stets hinter mir stehen und mich im gesamten Studium und in der Zeit dieser Arbeit unterstützt haben
%\end{itemize}
%%%%%%%%%%%%%%%%%%%%%%%%%%%%%%%%%%%%%%%%%%%%%%%%%%%%%%%%%%%%%%%%%%%%%%%%%%%%%%%%%%%%%%%%%%%%%%%%%%%%%%%%%%%%%%
%Literaturverzeichnis-----------------------------------------------------------------------------------------
\bibliographystyle{unsrt}
\bibliography{Literatur}
\addcontentsline{toc}{chapter}{Bibliography}%fügt das Literaturverzeichnis zum Inhaltsverzeichnis
\end{document}
