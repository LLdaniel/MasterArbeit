%%      --- TO DO --- 
%%    Die Frage ist, ob das jetzt noch eine Idee ist, den Detektoraufbau von der systematischen Teilchenbestimmungs-Seite aufzuziehen, quasi so, wie man es in der Masterclass lernt, mit einzelnen Spuren unterscheiden usw... --> siehe hierfür das Skript zusammenfassung ATLAS sowie https://cds.cern.ch/record/1096081/export/hx?ln=en und https://cds.cern.ch/record/1096081, oder ob das nicht dann doch zu viel wird
%%
%%    Danksagung --> noch überlegen, ob die nicht vielleicht doch auf Deutsch sein soll...
%-----------------------------------------------------------------------------------------------
%                   Gliederungsideen
%Experiment-Part: LHC, ATLAS, LQpairprod @pp collisions
%
%Software-Part: Datenprozessierung, MC Simulation, b-tagging, anti-kT(jet reconstruction), tau reconstruction, e+µ detektion? (Unterpunkt von Experimental setup->turning detector signals into physical objects)
%
%Theorie-Part: Standardmodell, beyond SM?(eigentlich klein hakten und eher versteckt von hinten rum in Theorie LQ einführen...), Theorie LQ(begründung, was es erkärt, so bisschen die beyond SM Schiene), 
%
%Ausgangspunkt, Forschungsfrage...
%
%Quasi eine Story erzählen: SM sehr erfolgreiche Theorie (->Theorie), aber auch nicht die ganze Geschichte (->beyond), das könnte LQ lösen (->LQ) dafür braucht man aber ein Nachweisgerät (->ATLAS, LHC...) und dass man an die Physik rankommt, braucht man Auswertung (->Datenprocessierung und btagging usw), was ist  der aktuelle Stand in der Vorschung, worauf wird sich hier in der Arbeit konzentriert (->ttbar tautau)
%
%-----------------------------------------------------------------------------------------------
% Seitenübersicht aus der Bachelor Arbeit                       vs                      Master Arbeit                                   Diff
% Einleitung              1.25                                                          
% Grundlagen              0.25 Vorgeplänkel                                             0.25                                            0
% Standardmodell          3                                                             
% mögliche Erweiterungen  2.5
% LQ                      2
% Experimentelle ...      0.25 Vorgeplänkel                                             0.25                                            0
% ATLAS                   3                                                             3.75                                            0.75
% Myonspectrometer        2.5                                                           2       (LHC)                                   -0.5
% LQ in pp coll           1
% Datenanalyse            0.25 Vorgeplänkel
% derzeitiger Stand       4
% Aufgabenstellung        1.5
% Datensätze              0.75
% Zusammenfassung         1
 
\documentclass[pdftex, a4paper, parskip=full*, open=any, BCOR=10mm, fontsize=12pt, DIV=12, headsepline, footsepline=true, footinclude=false, draft=false, captions=nooneline]{scrbook}  %pdftex,
%
%\usepackage[utf8]{inputenc}
\usepackage[T1]{fontenc}
\usepackage{lmodern}
\usepackage{amsmath, amsthm, amssymb}
\usepackage{mathtools}
\usepackage{graphicx}
\usepackage{array}
\usepackage{babelbib}
\usepackage{verbatim}
\usepackage{lscape}
\usepackage{textcomp}    %fuer aufrechte mu
\usepackage[english]{babel}
\usepackage{caption}%Ist Voraussetzung für das Subcaption package, welches die subfigures erlaubt
\usepackage{subcaption}%Für Einbindung von Bildern als Untergraphiken subfig(ure) sind veraltet!
\renewcommand*{\chapterformat}{%Chapter Design
  \enskip\mbox{\scalebox{2}{\thechapter\autodot}}}
\renewcommand\chapterlinesformat[3]{%
  \parbox[b]{\textwidth}{\textcolor{royalazure}{\hrulefill#2}}\par%
  #3\par\bigskip%
  \textcolor{royalazure}{\hrule}}
\RedeclareSectionCommand[afterskip=1.5\baselineskip]{chapter}
%\usepackage[numbers]{natbib}
\usepackage{hyperref}
\usepackage{url}
\usepackage{cancel} %for E/T miss durchgestrichen
%\usepackage{ziffer} %für Kommas bei Zahlen
\usepackage{multirow}
\usepackage[table]{xcolor}
%\usepackage{bibgerm}
%Aus dem Praktikum...
\usepackage{ae}                %macht schöneres ß
\usepackage[margin=10pt,font=small,labelfont=bf]{caption} %macht die Bildbeschriftungen richtig
\usepackage{epsfig}%Zur Einbindung von .eps: eps-->pdf
\usepackage{rotating}%für Querformat einer Tabelle 
\usepackage[separate-uncertainty=true, binary-units=true]{siunitx} %SI unit package mit binary units = bits, bytes
%\renewcommand{\figurename}{Abb.}
\setlength\parindent{1em}%Rückt am Absatzbeginn ein
%......ENDE
\renewcommand{\thefootnote}{\fnsymbol{footnote}}%Symbole für Fußnoten und keine default arabics
\makeatletter%Clear counter nach jedem Kapitel für die Fußnoten
\@addtoreset{footnote}{section}% ""
\makeatother% ""
\addtokomafont{caption}{\small}
\setkomafont{captionlabel}{\bfseries \sffamily}
%\bibliographystyle{gerplain} %Ist jetzt im Hauptdokument verbaut-->s. \bibliography{}
\definecolor{royalazure}{rgb}{0.0, 0.22, 0.66}
\renewcommand*\chapterheadstartvskip{\vspace*{-\topskip}}%Seitenrandanfang einstellen
\renewcommand*\chapterheadendvskip{%Seitenrandende einstellen
  \vspace*{1\baselineskip plus .1\baselineskip minus .167\baselineskip}}

\newcommand{\ATLAS}{\textsc{atlas}}%for small capitals, looks even nicer
\newcommand{\CERN}{\textsc{cern}}
\newcommand{\LHC}{\textsc{lhc}}
\newcommand{\ALICE}{\textsc{alice}}
\newcommand{\CMS}{\textsc{cms}}
\newcommand{\LINAC}{\textsc{linac}}
\newcommand{\SUSY}{\textsc{susy}}
\newcommand{\GUT}{\textsc{gut}}
\newcommand{\GEANT}{\textsc{geant}}
\newcommand{\ROOT}{\textsc{root}}
\newcommand{\aMCNLO}{M\textsc{adgraph5}\textunderscore aMC\textsc{nlo}}
\newcommand{\POWHEG}{P\textsc{owheg}-B\textsc{ox}}
\newcommand{\EvtGen}{E\textsc{vt}G\textsc{en}}
\newcommand{\Pythia}{P\textsc{ythia}8.2}
\newcommand{\NNPDFd}{\textsc{NNPDF}3.0\textsc{NLO}}
\newcommand{\NNPDFz}{\textsc{NNPDF}2.3\textsc{NLO}}
\hyphenation{
  %Aus-gangs-sig-nal Be-trach-tet er-folg-reich Über-gangs-strah-lungs-de-tek-tor
  po-si-tiv
}

% Keine "Schusterjungen"
	\clubpenalty = 9999
% Keine "Hurenkinder"
	\widowpenalty = 9999 \displaywidowpenalty = 9999

\begin{document}
\begin{titlepage}
  \vspace*{-7\baselineskip}
	\enlargethispage{100mm}
		\begin{center}
		\LARGE{\textbf{Master Thesis}\\}
		\vspace{3mm}	
		\textcolor{royalazure}{\noindent\rule{\textwidth}{3pt}}
		\huge{\textbf{Signal and background studies for scalar leptoquark pair production in the t$\bar{\textbf{t}}\,\mathbf{+\,2\tau}$ channel at the ATLAS experiment}\\}
		\vspace{3mm}
		\Large{Daniel Adlkofer\\}
		\textcolor{royalazure}{\noindent\rule{\textwidth}{3pt}}
		\vspace{3mm}
        \includegraphics[width=0.55\textwidth]{figures/neuSIEGEL.eps} \\
		\vspace{3mm}
		Supervisor \\
		\Large{Prof. Dr. Raimund Str\"{o}hmer\\}
               	\vspace{3mm}
                Advisor \\
    \Large{Dr. Mahsana Haleem\\}
               	\vspace{3mm}

		\vspace{5mm}
		December 2018\\
		  \noindent\hrulefill\\
		\vspace{3mm}
 		Lehrstuhl f\"{u}r Physik und ihre Didaktik\\
 		Physikalisches Institut\\
	    Julius-Maximilians-Universit\"{a}t W\"{u}rzburg
	\end{center}
\end{titlepage}
\cleardoubleoddemptypage

\tableofcontents
%%%%%%%%%%%%%%%%%%%%%%%%%%%%%%%%%%%%%%%%%%%%%%%%%%%%%%%%%%%%%%%%%%%%%%%%%%%%%%%%%%%%%%%%%%%%%%%%%%%
\chapter{XyZ}
%%%%%%%%%%%%%%%%%%%%%%%%%%%%%%%%%%%%%%%%%%%%%%%%%%%%%%%%%%%%%%%%%%%%%%%%%%%%%%%%%%%%%%%%%%%%%%%%%%%
\begin{table}[htbp]
		\centering
		\begin{tabular*}{\linewidth}{@{\extracolsep{\fill}}ccccc}
		\hline
		\hline
		\rule[-6pt]{0pt}{21pt} \textbf{sample}  & \textbf{t$\bar{\textbf{t}}$}  & \textbf{t$\bar{\textbf{t}}$H} & \textbf{LQ$_{\SI{500}{\giga\electronvolt}}$} & \textbf{LQ$_{\SI{1}{\tera\electronvolt}}$}
		\\
		\hline
		\rule[-7pt]{0pt}{23pt} selection  & reconstruction & reconstruction & reconstruction & reconstruction  
		\\ 
		\rule[-7pt]{0pt}{23pt}  & event yield & event yield & event yield & event yield 
		\\
		\hline
		\rule[-6pt]{0pt}{21pt} $\geq 2\,$b-jets   & $186\,395$ & $209$ & $152$ & $1.5$
		\\
		\rule[-6pt]{0pt}{21pt} $\geq 2\,$b-jets $+\geq1\,\tau$  & $505$ & $7$ & $94$ & $0.9$
		\\
		\rule[-6pt]{0pt}{21pt} $\geq 2\,$b-jets $+\geq2\,\tau$ & $1.7$ & $0.4$ & $27$ & $0.2$ 
		\\
		\hline
		\hline
		\end{tabular*}
		\caption[Event yield for the t$\bar{\text{t}}$, t$\bar{\text{t}}$H and the LQ samples.]{Event yield for different selections with tau leptons for the t$\bar{\text{t}}$, the t$\bar{\text{t}}$H and the LQ Monte Carlo sample. The luminosity accounts for $150\,\text{fb}^{-1}$.}
		\label{ttHttbarEvent}
	\end{table}
%
\begin{table}[htbp]
		\centering
		\begin{tabular*}{\linewidth}{@{\extracolsep{\fill}}ccc}
		\hline
		\hline
		\rule[-6pt]{0pt}{21pt} \textbf{sample}  & \textbf{t$\bar{\textbf{t}}$} & \textbf{t$\bar{\textbf{t}}$H}
		\\
		\hline
		\rule[-7pt]{0pt}{23pt} selection  & efficiency $\frac{\epsilon}{\%}$ & efficiency $\frac{\epsilon}{\%}$ 
		\\
		\hline
		\rule[-6pt]{0pt}{21pt} $\geq 2\,$b-jets & $26.52$ & $36.72$ 
		\\
		\rule[-6pt]{0pt}{21pt} $\geq 2\,$b-jets $+1\,\tau$  & $3.18$ & $8.83$ 
		\\
		\rule[-6pt]{0pt}{21pt} $\geq 2\,$b-jets $+2\,\tau$  & $1.41$ & $2.13$ 
		\\
		\hline
		\hline
		\end{tabular*}
		\caption[Efficiencies for the t$\bar{\text{t}}$ and the t$\bar{\text{t}}$H sample.]{Efficiencies for different selections with tau leptons for the t$\bar{\text{t}}$ and the t$\bar{\text{t}}$H Monte Carlo sample.}
		\label{ttHttbarEff}
	\end{table}
%
%
%
%
\begin{table}[htbp]
		\centering
		\begin{tabular*}{\linewidth}{@{\extracolsep{\fill}}cccccc}
		\hline
		\hline
		\rule[-6pt]{0pt}{21pt} \textbf{sample} & & \multicolumn{2}{c}{\textbf{t$\bar{\textbf{t}}$}}  & \multicolumn{2}{c}{\textbf{t$\bar{\textbf{t}}$H}} 
		\\
		\hline
		\rule[-7pt]{0pt}{23pt} \multirow{2}{*}{selection} & reference & reconstruction & truth & reconstruction & truth  
		\\ 
		\rule[-7pt]{0pt}{23pt} & selection & ratio $\frac{r}{\%}$ & ratio $\frac{r}{\%}$ & ratio $\frac{r}{\%}$ & ratio $\frac{r}{\%}$ 
		\\
		\hline
		\rule[-6pt]{0pt}{21pt} $\geq 2\,$b-jets $+1\,\tau$ & $\geq 2\,$b-jets & $0.28$ & $2.35$ & $3.43$ & $14.26$
		\\
		\rule[-6pt]{0pt}{21pt} $\geq 2\,$b-jets $+2\,\tau$ & $\geq 2\,$b-jets & $0.0011$ & $0.020$ & $0.24$ & $4.11$ 
		\\
		\hline
		\hline
		\end{tabular*}
		\caption[Ratios for the t$\bar{\text{t}}$ and the t$\bar{\text{t}}$H sample.]{Ratios for different selections with tau leptons for the t$\bar{\text{t}}$ and the t$\bar{\text{t}}$H Monte Carlo sample.}
		\label{ttHttbarRatio}
	\end{table}
%	
%
%	
%	
	\begin{table}[htbp]
		\centering
		\begin{tabular*}{\linewidth}{@{\extracolsep{\fill}}ccccc}
		\hline
		\hline
		\rule[-6pt]{0pt}{21pt} \textbf{sample}  & \multicolumn{2}{c}{\textbf{t$\bar{\textbf{t}}$}}  & \multicolumn{2}{c}{\textbf{t$\bar{\textbf{t}}$H}} 
		\\
		\hline
		\rule[-7pt]{0pt}{23pt} \multirow{2}{*}{selection}  & numerator      & denominator & numerator      & denominator
		\\ 
		\rule[-7pt]{0pt}{23pt}                             & event yield    & event yield & event yield    & event yield 
		\\
		\hline
		\rule[-6pt]{0pt}{21pt} truth matching for tau      & $63$            & $13723$      & $5590$        & $21610$
		\\
		\rule[-6pt]{0pt}{21pt} efficiency                  & \multicolumn{2}{c}{$0.46\%$}    & \multicolumn{2}{c}{$25.9\%$}
		\\
		\hline
		\rule[-6pt]{0pt}{21pt} tau from H$^0$, W$^{\pm}$, Z$^0$& $0$        & $0$         & $4859$          & $11988$
		\\
		\rule[-6pt]{0pt}{21pt} efficiency                  & \multicolumn{2}{c}{-}   & \multicolumn{2}{c}{$40.5\%$}
		\\
		\hline
		\rule[-6pt]{0pt}{21pt} tau from B-mesons           & $63$            & $13722$      & $20$            & $7416$ 
		\\
		\rule[-6pt]{0pt}{21pt} efficiency                  & \multicolumn{2}{c}{$0.46\%$}   & \multicolumn{2}{c}{$0.27\%$}
		\\
		\hline
		\rule[-6pt]{0pt}{21pt} tau within a jet            & $8440$         & $3776952$      & $18511$         & $20327225$ 
		\\
		\rule[-6pt]{0pt}{21pt} efficiency                  & \multicolumn{2}{c}{$0.22\%$}   & \multicolumn{2}{c}{$0.091\%$}
		\\
		\hline
		\rule[-6pt]{0pt}{21pt} tau within a b-jet          & $6098$        & $2658379$      & $2317$         & $1208924$ 
		\\
		\rule[-6pt]{0pt}{21pt} efficiency                  & \multicolumn{2}{c}{$0.23\%$}   & \multicolumn{2}{c}{$0.19\%$}
		\\
		\hline
		\hline
		\end{tabular*}
		\caption[Event yield for the t$\bar{\text{t}}$ and the t$\bar{\text{t}}$H sample.]{Event yield for different selections with tau leptons for the t$\bar{\text{t}}$ and the t$\bar{\text{t}}$H Monte Carlo sample. The luminosity accounts for $36.1\,\text{fb}^{-1}$.}
		\label{ttHttbarEventTruthMatching}
	\end{table}
%
%
%
%
	\begin{table}[htbp]
		\centering
		\begin{tabular*}{\linewidth}{@{\extracolsep{\fill}}ccccc}
		\hline
		\hline
		\rule[-6pt]{0pt}{21pt} \textbf{sample}  & \multicolumn{2}{c}{\textbf{LQ${_{\mathbf{\SI{500}{\giga\electronvolt}}}}$}}  & \multicolumn{2}{c}{\textbf{LQ${_{\SI{1}{\tera\electronvolt}}}$}} 
		\\
		\hline
		\rule[-7pt]{0pt}{23pt} \multirow{2}{*}{selection}  & numerator      & denominator & numerator      & denominator
		\\ 
		\rule[-7pt]{0pt}{23pt}                             & event yield    & event yield & event yield    & event yield 
		\\
		\hline
		\rule[-6pt]{0pt}{21pt} truth matching for tau      & $2604$            & $5362$      & $2263$        & $5055$
		\\
		\rule[-6pt]{0pt}{21pt} efficiency                  & \multicolumn{2}{c}{$48.6\%$}    & \multicolumn{2}{c}{$44.8\%$}
		\\
		\hline
		\rule[-6pt]{0pt}{21pt} tau from H$^0$, W$^{\pm}$, Z$^0$& $95$        & $340$         & $82$          & $461$
		\\
		\rule[-6pt]{0pt}{21pt} efficiency                  & \multicolumn{2}{c}{$27.9\%$}   & \multicolumn{2}{c}{$17.8\%$}
		\\
		\hline
		\rule[-6pt]{0pt}{21pt} tau from B-mesons           & $0$            & $183$      & $0$            & $200$ 
		\\
		\rule[-6pt]{0pt}{21pt} efficiency                  & \multicolumn{2}{c}{$0.0\%$}   & \multicolumn{2}{c}{$0.0\%$}
		\\
		\hline
		\rule[-6pt]{0pt}{21pt} tau from LQ                 & $1744$            & $3286$      & $1057$            & $2022$ 
		\\
		\rule[-6pt]{0pt}{21pt} efficiency                  & \multicolumn{2}{c}{$53.1\%$}   & \multicolumn{2}{c}{$52.3\%$}
		\\
		\hline
		\rule[-6pt]{0pt}{21pt} tau within a jet            & $7232$         & $55208$      & $7011$         & $63671$ 
		\\
		\rule[-6pt]{0pt}{21pt} efficiency                  & \multicolumn{2}{c}{$13.1\%$}   & \multicolumn{2}{c}{$11.0\%$}
		\\
		\hline
		\rule[-6pt]{0pt}{21pt} tau within a b-jet          & $2317$        & $1208924$      & $6098$         & $2658379$ 
		\\
		\rule[-6pt]{0pt}{21pt} efficiency                  & \multicolumn{2}{c}{$0.45\%$}   & \multicolumn{2}{c}{$0.23\%$}
		\\
		\hline
		\hline
		\end{tabular*}
		\caption[]{}
		\label{LQEventTruthMatching}
	\end{table}
%%%%%%%%%%%%%%%%%%%%%%%%%%%%%%%%%%%%%%%%%%%%%%%%%%%%%%%%%%%%%%%%%%%%%%%%%%%%%%%%%%%%%%%%%%%%%%%%%%%%%%%%%%%%%%%%%%%%%%%%%%%
\chapter{Introduction}
%einen gescheiten Einleitungsgedanken überlegen! Ja nicht Higgs... :see_no_evil:
%Yukawa Kopplung vielleicht? --> ich bin mir nicht mehr sicher, aber vielleicht habe ich da einen Artikel gelesen (News CERN oder so) wo es um die noch bessere Messung der Yukawa Kopplung ging...
%%%%%%%%%%%%%%%%%%%%%%%%%%%%%%%%%%%%%%%%%%%%%%%%%%%%%%%%%%%%%%%%%%%%%%%%%%%%%%%%%%%%%%%%%%%%%%%%%%%%%%%%%%%%%%%%%%%%%%%%%%%%%
\chapter{Theoretical background for the search for scalar leptoquarks}\label{theory}
This chapter describes some theoretical foundations required for the search for scalar leptoquarks including the successful Standard Model of elementary particle physics evolved from the symbiosis of experimental achievements and theoretical milestones. Besides its success some issues still remain unsolved and could be a hint to physics beyond the Standard Model, giving space to introduce the Leptoquark Model as one possible extension. 
%%%%%%%%%%%%%%%%%%%%%%%%%%%%%%%%%%%%%%%%%%%%%%%%%%%%%%%%%%%%%%%%%%%%%%%%%%%%%%%%%%%%%%%%%%%%%%%%%%%%%%%%%%%%%%%%%%%%%%%%
\section{The Standard Model of particle physics}\label{SM}
A remarkable development for understanding nature is the Standard Model of particle physics, embracing physics at the most fundamental level. This quantum field theory, incorporating the conceptual frameworks of special relativity and quantum mechanics, describes the constituents of matter and the laws governing their interactions. \cite{Mann} Despite its success of being the best theory so far capable of explaining the observed results within its domain in agreement with empirical data, it seems not to be the complete story. There are still many puzzles left which are not described by the Standard Model. That circumstance keeps physicists well motivated to gain further progress and to push the frontiers of our understanding. \cite{Nair}\par
One of the most important concept in physics is that of symmetries, because they are deeply connected with conservation of laws, following Noether`s Theorem. Physical properties can appear in form of an invariant under symmetry transformation, leaving that property unchanged, or as covariant, changing their property induced by the symmetry transformation. %Continuous symmetries yield additive laws whereas discrete symmetries result in multiplicative laws.
\begin{table}[htbp]
		\centering
		\begin{tabular*}{\linewidth}{@{\extracolsep{\fill}}cccc}
		\hline
		\hline
		\rule[-6pt]{0pt}{21pt}\textbf{group}  & \textbf{defining property} & \textbf{application} 
		\\
		\hline
		\rule[-6pt]{0pt}{21pt} $U(n)$ &  $n\times n\,$ unitary ($U^\dagger U=1$)	& $U(1)$ electromagnetism
		\\\\
                \rule[-6pt]{0pt}{21pt} \multirow{2}{*}{ $SU(n)$ } &  $n\times n\,$ unitary ($U^\dagger U=1$) 	& $SU(2)$ weak interactions
		\\
                \rule[-6pt]{0pt}{21pt}  & with $\text{det}U=1$  &  $SU(3)$ strong interactions
		\\\\
                \rule[-6pt]{0pt}{21pt} \multirow{2}{*}{$SO(n)$} & $n\times n\,$ orthogonal ($O^\intercal O=1$)  &  $SO(3)$ rotations
		\\
                \rule[-6pt]{0pt}{21pt}  &  with $\text{det}O=1$ &  $SO(3,1)$ Lorentz transformations
		\\
		\hline
		\hline
		\end{tabular*}
		\caption[Lie symmetry groups of the Standard Model.]{Lie symmetry groups for the gauge interactions of the Standard Model \cite{Mann}.}
		\label{sym}
	\end{table}
Foundational symmetries of particle include space translation symmetry and hence the conservation of momentum, rotational invariance and hence conservation of angular momentum as well as time translation invariance leading to energy conservation. The proper mathematical description for fundamental symmetries involves group theory. In case of particle physics almost all groups are Lie groups $\mathcal{G}$, that is a set of objects $\{g_i\}$, which can be combined with a binary operation and has four basic properties: closure, identity, inverse element and associativity. For Lie groups additionally the group elements are continuous and differentiable function of some finite set of parameter $\theta_{a}$:
\begin{align}
                g=g(\theta_1, \ldots, \theta_N)=\exp\left[i\theta_a\mathbf{T}^a\right]=\exp\left[i\vec{\theta}\vec{\mathbf{T}}\right]\qquad \text{with}\quad a=1,\ldots,N
\end{align}
Here $\mathbf{T}^a$ are the generators of the group from which all elements of the group can be created. The irreducible representatives of a group can be written as complex matrices\footnote{Irreducible means that not all representing matrices of the group can be decomposed into block-diagonal form simultaneously \cite{Mann}.}, acting on the wavefunction of the particles and on charges as well as on space-time coordinates. \cite{Mann}
The local symmetry $SU(3)_c\times SU(2)_L\times U(1)_Y$ summarizes the gauge interactions of the Standard Model (see table \ref{sym}). Here $c$ indicates the strong force, $L$ the left handed chirality of the weak regime and $Y=B+s$ the hypercharge calculated from baryon number $B$ and strangeness $s$. \cite{PhysTeV} Besides the continuous symmetries above also important discrete symmetries exist in the Standardmodel like parity $P$, referring to the transformation $\vec{x}\rightarrow-\vec{x}$, time reversal $T$, referring to $t\rightarrow-t$, and charge conjugation $C$, corresponding to the exchange of a particle with its anti-particle. The weak force breaks $P$ and $C$, but not the product of $CPT$. \cite{Nair}\par
Matter and its interactions can be described by two basic types of particles and the fundamental forces, i.e. the electromagnetic force, the weak and the strong force and gravity. Fermionic particles, following Fermi-Dirac statistics, make up matter, whereas bosons, following Bose-Einstein statistics, are acting as mediators of the fundamental forces. \cite{Cottingham}\cite{Griffiths}\newline
The fermions can be further categorized into $6$ leptons $l$ characterized through lepton quantum number $L_l$ and $6$ quarks $q$ characterized through baryon quantum number $B$ together with their anti-particles ($\bar{l}$ respectively $\bar{q}$). The only difference between particle and anti-particle is contrary electrical charge and contrary lepton respectively baryon number. Leptons occur in three generations with different flavour -- electron (e), muon ($\mu$) and tauon ($\tau$) -- and can carry electrical charge $Q=\pm e$ in units of the elementary charge or electrical neutral neutrinos $\nu_l$. \cite{Griffiths}\newline
Table \ref{SMtable} shows the leptons with selected properties like mass. Various neutrino oscillation experiments confirm that neutrinos have non-zero masses, although the Standard Model does not predict neutrino masses \cite{Kajita}\cite{McDonald}. The flavor states $\nu_\alpha$ with $\alpha=e,\mu,\tau$ are quantum entangled with the mass states $\nu_i$ where $i=1,2,3$ described by an unitary matrix $U_{\alpha i}$ \cite{mixing}. Because of the absence of the right chirality spinor components, the mass term $\bar{\Psi}_L\Psi_R+\bar{\Psi}_R\Psi_L$ cannot be formed in case of neutrinos. One possible solution is to introduce very massive\footnote{So massive that it is beyond todays oberservable mass limit \cite{Nair}.} right-chiral neutrinos, giving them the differnt masses with a Majorana mass term. This corresponds to the prediction of the seesaw meachnism. \cite{Nair}\newline
\begin{table}[htbp]
		\centering
		\begin{tabular*}{\linewidth}{@{\extracolsep{\fill}}cccccc}
		\hline
		\hline
		\rule[-6pt]{0pt}{21pt}\textbf{leptons}  & & & &
		\\
		\hline
		\rule[-7pt]{0pt}{23pt} $l$ & $L_l$ & $B$ & $Q/e$ & $m$/$\frac{\SI{}{\giga\electronvolt}}{\SI{}{\text{\ensuremath{c}\squared}}}$ & Spin $S_z/\hbar$
		\\
		\hline
		\rule[-6pt]{0pt}{21pt} e$^-$ & \(L_e=1\) & $0$ & $-1$ & $0.511$ & $\frac{1}{2}$
		\\
		\rule[-6pt]{0pt}{21pt}$\nu_e$&  \(L_e=1\)	& $0$ & $0$ & $<2\cdot10^{-6}$ & $\frac{1}{2}$
		\\
		\rule[-6pt]{0pt}{21pt} $\mu^-$ & \(L_{\mu}=1\) & $0$ & $-1$ & $106$ & $\frac{1}{2}$
		\\
		\rule[-6pt]{0pt}{21pt}$\nu_{\mu}$ &  \(L_{\mu}=1\)	& $0$ & $0$ & $<2\cdot10^{-6}$ & $\frac{1}{2}$
		\\
		\rule[-6pt]{0pt}{21pt} $\tau^-$ & \(L_{\tau}=1\) & $0$ & $-1$ & $1.78\cdot10^3$ & $\frac{1}{2}$
		\\
		\rule[-6pt]{0pt}{21pt}$\nu_{\tau}$&  \(L_{\tau}=1\)	& $0$ & $0$ & $<2\cdot10^{-6}$ & $\frac{1}{2}$
		\\
		\rule[-6pt]{0pt}{21pt}  & & & &
		\\
		\rule[-6pt]{0pt}{21pt}\textbf{quarks}  & & & &
		\\
		\hline
		\rule[-7pt]{0pt}{23pt} $q$ & \(L_l\) & $B$ & $Q/e$ & $m$/$\frac{\SI{}{\giga\electronvolt}}{\SI{}{\text{\ensuremath{c}\squared}}}$ & Spin $S_z/\hbar$
		\\
		\hline
		\rule[-6pt]{0pt}{21pt} u (up) &  $0$	& $\frac{1}{3}$ & $\frac{2}{3}$ & $2.2$ & $\frac{1}{2}$
		\\
		\rule[-6pt]{0pt}{21pt} d (down) &  $0$	& $\frac{1}{3}$ & $-\frac{1}{3}$ & $4.7$ & $\frac{1}{2}$
		\\
		\rule[-6pt]{0pt}{21pt} c (charm) &  $0$	& $\frac{1}{3}$ & $\frac{2}{3}$ & $1.3\cdot10^3$ & $\frac{1}{2}$
		\\
		\rule[-6pt]{0pt}{21pt} s (strange) &  $0$	& $\frac{1}{3}$ & $-\frac{1}{3}$ & $95$ & $\frac{1}{2}$
		\\
		\rule[-6pt]{0pt}{21pt} t (top) &  $0$	& $\frac{1}{3}$ & $\frac{2}{3}$ & $17\cdot10^{4}$ & $\frac{1}{2}$
		\\
		\rule[-6pt]{0pt}{21pt} b (bottom) &  $0$	& $\frac{1}{3}$ & $-\frac{1}{3}$ & $4.2\cdot10^{3}$ & $\frac{1}{2}$
		\\
		\rule[-6pt]{0pt}{21pt}  & & & &
		\\
		\rule[-6pt]{0pt}{21pt} \textbf{gauge bosons} & & & &
		\\
		\hline
		\rule[-7pt]{0pt}{23pt} boson & \(L_l\) & $B$ & $Q/e$ & $m$/$\frac{\SI{}{\giga\electronvolt}}{\SI{}{\text{\ensuremath{c}\squared}}}$ & Spin $S_z/\hbar$
		\\
		\hline
		\rule[-6pt]{0pt}{21pt} $\gamma$ &  $0$	& $0$ & $0$ & $0$ & $1$
		\\
		\rule[-6pt]{0pt}{21pt} Z$^0$ &  $0$	& $0$ & $0$ & $91.2$ & $1$
		\\
		\rule[-6pt]{0pt}{21pt} W$^-$ &  $0$	& $0$ & $-1$ & $80.4$ & $1$
		\\
		\rule[-6pt]{0pt}{21pt} W$^+$ &  $0$	& $0$ & $+1$ & $80.4$ & $1$
		\\
		\rule[-6pt]{0pt}{21pt} g &  $0$	& $0$ & $0$ & $0$ & $1$
		\\
		\rule[-6pt]{0pt}{21pt} H$^0$ &  $0$	& $0$ & $0$ & $125$ & $0$
		\\
		\hline
		\hline
		\end{tabular*}
		\caption[Overview of elementary particles with some selected properties.]{Overview of leptons $l$, quarks $q$ and gauge bosons as mediators of the forces with some selected properties and quantum numbers like electrical charge $Q$, mass $m$, lepton number $L_l$ and baryon number $B$. \cite{Nair}\cite{PhysRevD}. Anti-particles are not shown due to the only difference in opposite electrical charge and lepton/baryon number.}
		\label{SMtable}
	\end{table}
The quarks also occur in three generations and carry electrical charge $Q=\pm e$ in units of the elementary charge as well as color charge. Possible color charges are red, green and blue and the additional anti-colors indicating that quarks are interacting with the strong force among others. \cite{Griffiths} Quarks only occur confined in color-neutral compound systems called hadrons. Baryons are three-quark states with baryon number $1$ and mesons are quark-anti-quark states, having a baryon number of $0$. \cite{Cottingham} The reason for quark confinement can be found in the potential $V(r)$ between quarks and anti-quarks depending on distance $r$. The potential has the shape $V(r)\propto\frac{-1}{r}+\text{const}\cdot r$. When seperating a quark-anti-quark pair, additional potential energy is supplied, which can exceed at distance $R$ the potential $V(R)>2m_q$ for more than two quark masses. Quantum fluctuations result in the origin of a new quark pair in between, now having two pairs again externally color-neutral. \cite{Nair} Table \ref{SMtable} shows the different quark flavor up, down, charm, strange, top and bottom with some characteristic properties. \newline
The bosons shown in table \ref{SMtable} are the quanta of the fundamental forces \cite{Cottingham}:
\begin{itemize}
\item{The photon $\gamma$ is the mediator of the electromagnetic force.}
\item{Three mediators Z$^0$, W$^+$ and W$^-$ for the weak force.}
\item{$8$ colored gluons as mediators for the strong force.}
\item{The Higgs boson H$^0$ as quantum of the Higgs field, providing the masses for the elementary particles.}
\end{itemize}
The mediator of a quantum field theory of gravity would be the graviton, altough this is hypothetically at the current state of research. A second aspect is that on elementary particle scales gravity is insignificant compared\footnote{Relative strength of gravity compared to the weak interaction is $10^{-35}$ \cite{Mann}.} to all other forces and is therefore not originally considered in the Standard Model. \cite{Cottingham}\par
%
\begin{figure}[htbp]                                 
 \begin{center}                                       
  \includegraphics[width=0.5\linewidth]{figures/SMpic.pdf} 
   \caption[Overview of the Standard Model.]{The Standard Model with its fermions and bosons and the involved interactions. The solid blue line indicates which particles interact with each other. Loops include self-interaction. \cite{SMpic}}
  \label{SMpic}                                     
 \end{center}
\end{figure}
%
Figure \ref{SMpic} summarizes the picture of the Standard Model with its fermions and bosons. The lines indicate which particles interact with each other trough the mediators, including the self-interaction.
%Gedanken zum SM
%strong (n und p of same duplet) and weak isospin (chirality)
%%%%%%%%%%%%%%%%%%%%%%%%%%%%%%%%%%%%%%%%%%%%%%%%%%%%%%%%%%%%%%%%%%%%%%%%%%%%%%%%%%%%%%%%%%%%%%%%%%%%%%%%%%%%%%%%%%%%%%%%%%%%
\section{Beyond the scope of The Standard Model}\label{beyondSM}
%GUT und B anomalies explanation within LQ models
%%%%%%%%%%%%%%%%%%%%%%%%%%%%%%%%%%%%%%%%%%%%%%%%%%%%%%%%%%%%%%%%%%%%%%%%%%%%%%%%%%%%%%%%%%%%%%%%%%%%%%%%%%%%%%%%%%%%%%%%%%%%
\section{Leptoquark Models}\label{LQmodels}
%%%%%%%%%%%%%%%%%%%%%%%%%%%%%%%%%%%%%%%%%%%%%%%%%%%%%%%%%%%%%%%%%%%%%%%%%%%%%%%%%%%%%%%%%%%%%%%%%%%%%%%%%%%%%%%%%%%%%%%%%%%%%
\chapter{Experimental setup for the search for scalar leptoquarks}\label{experiment}
For the search for scalar leptoquarks the {\ATLAS} detector at the Large Hadron Collider ({\LHC}) is used as experimental setup, which will be described within this chapter. The general setting of the proton-proton collider located at the {\CERN} research center is the topic of section \ref{LHC}. The particle detection of the resulting collision events will take place in the {\ATLAS} detector with its different specialized components (section \ref{ATLAS}). Section \ref{LQpp} addresses the leptoquark pair production in proton-proton collisions.  
%%%%%%%%%%%%%%%%%%%%%%%%%%%%%%%%%%%%%%%%%%%%%%%%%%%%%%%%%%%%%%%%%%%%%%%%%%%%%%%%
\section{The Large Hadron Collider accelerator complex}\label{LHC}
The research center {\CERN} (Conseil Europ\'{e}en pour la Recherche Nucl\'{e}aire) was founded in $1954$ near Geneva, Switzerland to become a major European joint venture on elementary particle physics. Currently $22$ member states are participating in that large-scale project with the ambition to probe the essential constituents of nature and the fundamental forces acting between them. \cite{CERNabout}\par
%
\begin{figure}[htbp]                                 
 \begin{center}                                       
  \includegraphics[width=0.8\linewidth]{figures/CERNKomplex2.jpg} 
   \caption[Schematic of the {\CERN} accelerator complex.]{Schematic of the {\CERN} accelerator complex with its different stages and few experiments like {\ATLAS} located at one crossing point for protons. \cite{CERNKomplex}}
  \label{complex}                                     
 \end{center}
\end{figure}
%
In the accelerator complex protons reach energies of $\SI{6.5}{\tera\electronvolt}$ by going through different accelerator stages and are brought to collisions at defined interaction sites in time intervals of $\SI{25}{\nano\second}$. Particle detectors then register signatures of the resulting collision events and the analysis of newly created particles gives insight to the nature of elementary particle physics.\newline 
Figure \ref{complex} shows the different acceleration stages. Starting from the injection, protons will gain a kinetic energy of $\SI{50}{\mega\electronvolt}$ in the linear accelerator {\LINAC}2 and will be further transferred to the Proton Synchrotron Booster ($\SI{1.4}{\giga\electronvolt}$), the Proton Synchrotron ($\SI{25}{\giga\electronvolt}$), the Super Proton Synchrotron ($\SI{450}{\giga\electronvolt}$) and finally to the {\LHC} ring with its $\SI{26.7}{\kilo\meter}$ circumference. \cite{CERNabout}\newline
The {\LHC} is designed as two-ring proton-proton collider. Conditions for a stable proton beam are diverse, including high vacua of $\SI{E-10}{\milli\bar}$ to $\SI{E-11}{\milli\bar}$ and temperatures of $\SI{1.9}{\kelvin}$ for the superconducting NbTi-magnets of the accelerator. \cite{LHCJINST}\par   
%hier dann weiter ins Detail wie LHC JINST #######################
%Die Idee ist eigentlich eine genauere Beschreibung der Komponenten, allerdings ist JINST schon wieder so tief... So überblicksmäßiger wäre viel bessser, bei dem man noch im Blick behalten kann, um was es eigentlich geht...
Different experiments like \ALICE\cite{ALICE}, {{\LHC}}b\cite{LHCb} are located at the {\LHC} due to the variety of research questions. But the subject of interest in this work lies in the high luminosity experiment {\ATLAS}, which is specialized for proton-proton collisions, like its counterpart \CMS\cite{CMS}. Main tasks of {\ATLAS} are more precise measurements of the SM (see chapter \ref{SM}), better understanding Quantum Chromo Dynamics (QCD) and search for supersymmetric models, and new physics, among others. With the {\LHC} production of $10^9$ inelastic events per second, up to $23$ simultaneously events at dominating high QCD cross sections reqiure a powerful detector that is capable of recognizing the characteristic signatures. These circumstances make up the demands for {\ATLAS}, including fast electronic elements, high detector granularity, handling high particles fluxes and reducing overlapping events at a large acceptance and coverage region. \cite{ATLASJINST}       
%AußerdeM: Hinweis auf high Lumi LHC als Art mini-Ausblick in diesem Kapitel https://arxiv.org/pdf/1705.08830.pdf, https://cds.cern.ch/record/1711887/files/ATL-COM-UPGRADE-2014-014.pdf ################
%%%%%%%%%%%%%%%%%%%%%%%%%%%%%%%%%%%%%%%%%%%%%%%%%%%%%%%%%%%%%%%%%%%%%%%%%%%%%%%%%%%%%%%%%%%%%%%%%%%%%
\section{The ATLAS detector at the LHC}\label{ATLAS}
One of the general purpose detector for proton-proton collisions is the {\ATLAS} detector. This $\SI{25}{\meter}$ tall detector is located at one interaction point of the {\LHC} where bunches, consisting of approximately $\SI{E11}{}$ protons, collide at a rate of $\SI{40}{\mega\hertz}$ \cite{ATLASJINST}. The number of particles encountered per time is given by \cite{Perkins}
\begin{align}
                        \dot{N}=\mathcal{L}\sigma
\end{align}
with the cross section $\sigma$ for the present event and the instantantaneous luminosity $\mathcal{L}$. Given a measure for the number of collisions per unit time the instantaneous luminosity can be introduced and is often used as key parameter in collider physics \cite{LHCJINST}.
\begin{align}
                        \mathcal{L}=\frac{N_bn_bf_{\text{rev}}\gamma_r}{4\pi\epsilon_n\beta^*}F
\label{Lumi}
\end{align}
Where $N_b$ is the number of particles per bunch, $n_b$ the number of bunches per beam, $f_{\text{rev}}$ the rotational frequency, $\gamma_r$ the Lorentz factor, $\epsilon_n$ the normalized transverse beam emittance, $\beta^*$ the betatron function at the collision point and $F$ respects the geometric luminosity reduction factor due to the crossing angle at the collision point. The luminosity of {\ATLAS} exceeded the design luminosity of $\mathcal{L}=\SI{2.05E34}{\per\square\centi\meter\per\second}$ by a factor of $2.05$ on the $2^{\text{nd}}$ of November 2017, emphasizing the great success over the years \cite{designLumiExceeded}.\par 
%Erklärung für Lumi auch aus Perkins? ##########
The aspiration to be sensitive to the great variety of particles governed by the fundamental forces (see chapter \ref{SM}) influenced the detector design accordingly. The layered structure reflects the fact that
%Motivation eher sogar von der Seite, wieso so gebaut, wie er ist --> Teilchen detektieren (Verweis auf SM Kaptiel)-> Schichtaufbau-> etc... #######
The basic structure of {\ATLAS} is shown in figure \ref{ATLASandCoordinate} with its different sub-detector systems together with the convention for the used coordinate system.  %Koordinatensystem
%
\begin{figure}
 \centering
  \begin{subfigure}[c]{0.95\textwidth}
   \includegraphics[width=\textwidth]{figures/ATLASDesign.png}
    \subcaption[The structure of the {\ATLAS} detecor and its sub-systems.]{The layered structure of the {\ATLAS} Detector at the {\LHC} with its sub-systems Inner Detector, Calorimeter, magnets and Muon Spectrometer \cite{ATLASJINST}.}
   \label{ATLASDesign}
  \end{subfigure}
  \begin{subfigure}[c]{0.95\textwidth}
   \includegraphics[width=1\textwidth]{figures/ATLASDesignCoordinate.png}
   \subcaption[Definition of the global {\ATLAS} coordinate system.]{The global {\ATLAS} coordinate system formulated in cylindric coordinates with the $z$-axis parallel to the beam line and the transverse plane defined through azimuthal angle $\phi$ and pseudorapidity $\eta$. Based on \cite{ATLASJINST}.}
   \label{Coordinate}
  \end{subfigure}
 \caption[Structure of the {\ATLAS} detector and the used coordinate system.]{Structure of the {\ATLAS} detector and the used coordinate system.}
 \label{ATLASandCoordinate}
\end{figure}
%
The nominal interaction point acts as origin of the coordinate system, where the $z$-axis follows the beam line counterclockwise. Perpendicular to the $z$ axis lies the transverse $x$-$y$-plane usually described through the azimuthal angle $\phi$. The positive $x$-axis points towards the center of the {\LHC}. The cylindric symmetry of the detector suggests a cylindric coordinate system with the angle $\theta$ starting from the beamline. \cite{ATLASJINST} Since the polar angle is not a Lorentz invariant quantity, it is useful to describe the position in terms of rapidity \cite{LHCJINST} $w=\frac12\ln{\frac{E+p_zc}{E-p_zc}}$ in that highly relativistic regime. In the limit of large momenta, i.e. $|\mathbf{p}|c\approx E$, the rapidity coincides with the pseudorapidity formulated as \cite{ChinaPseudorapidityBook}
\begin{align}
                        \eta=-\ln{\tan\frac{\theta}{2}}\text{.}
\label{pseudorapidity}
\end{align}
This variable has only the polar angle as dependence and is therefore the adequate quantity in the context of collision experiments, where usually the angle $\theta$ from the beamline is measured. \cite{ChinaPseudorapidityBook}\par      
%Aufbau Idee: auch so ein Walkthrough durch die einzelnen Komponenten wie bei der Bachelorarbeit
%einzelne Komponenten + Trigger(ref auf Datenauswertung, Software part im Text dann)
\textbf{The magnet configuration} includes a superconducting solenoid with a field strength of $\SI{2}{\tesla}$ sourrounding the inner detector as well as three large superconducting toroid magnets composed in an eight-fold azimuthal symmetry around the calorimeter. The barrel toroid magnet delivers a field strength of $\SI{0.5}{\tesla}$ and in the end-cap a field of $\SI{1}{\tesla}$ is present. \cite{ATLASJINST}\newline%Bending particle path -->identification... ######
\textbf{The inner detector} is responsible for pattern recognition, momentum and vertex measurements and electrically charged particle identification which is achieved with a combination of semiconductor pixel and microstrip trackers (SCT). The Insertable B-Layer (IBL) is the innermost layer of the pixel detectors at a radius of $\SI{3.3}{\centi\meter}$ away from the beam line. Additional straw tube tracking detectors are sensitive to transistion radiation (TRT) in the outer part that are responsible for high vertex and momentum resolution. The $R-\phi$ segmented pixel detectors are of size $50\times \SI{400}{\square\micro\meter}$ and the SCTs with its $8$ strip layers cover together a range of $|\eta|<2.5$. Typically $36$ hits per track are provided by the $\SI{4}{\milli\meter}$ straw tubes of the TRTs, which cover the range $|\eta|\leq 2.0$. \cite{IBL}\cite{ATLASJINST}\newline %Unterscheidung teilchen anhand der Übergangsstrahlung ####################
Liquid argon electromagnetic sampling \textbf{calorimeters} with high granularity allow an excellent energy measurement for electrons and photons. It has a total thickness of more than $22$ radiation lengths $X_0$ in the barrel region ($|\eta|<1.475$) and more than $24X_0$ in the end-cap region ($1.375<|\eta|<3.2$). For hadronic energy measurements a scintillator-tile calorimeter covering $|\eta|<1.7$ is in operation. It is a sampling calorimeter and uses steel as absorber material and scintillating tiles as active material in conjunction with wavelength shifting fibres. Further LAr technology is used for hadronic particles in the outer pseudorapidity range up to $|\eta|=3.2$. Here copper plates provide the absorber material. The forward calorimeters extend the coverage for hadronic and electromagnetic energy measurements to $|\eta|=4.9$ and are $10X_0$ deep. \cite{ATLASJINST}\newline%Energiemessung: provide god res for high energy jets###################
The \textbf{muon system} is suited in the outer layer of {\ATLAS} and provides as independent system resolution for high energy muon tracks with three layered precision chambers. This is possible because of the air-cored toroid magnet system including one barrel and two end-cap magnets generating strong bending power in a large volume and delivering  a mostly perpendicular magnetic field regarding the muon trajectories. The bending power $\int{\vec{B}d\vec{l}}$ along the track of the muon $d\vec{l}$ reaches $\SI{1.5}{\tesla\meter}$ to $\SI{5.5}{\tesla\meter}$ in the range $|\eta|<1.4$ (barrel) and up to $\SI{7.5}{\tesla\meter}$ (end-cap). The precision chambers are Monitored Drift Tubes (MDT) and in the larger pseudorapidity range Chathode Strip Chambers (CSC) which are multiwire proportional chambers. Due to the fact that the overall performance depends crucially on the alignment of the muon detectors with respect to each other and in respect to the Inner Detector, MDTs are equipped with a optical monitoring system with $1200$ sensors. Resistive Plate Chambers (RPC) and Thin Gap Chambers (TGC) are the constituents of the muon trigger system. \cite{ATLASJINST} \newline %Myonen kommen überall durch-->hier detektieren + Problem der Spurkrümmung für high pt--> eigenes system für die genauere myon-messung ############
Due to technology and resource limitations the data recording rate has to be reduced from $\SI{40}{\mega\hertz}$ to $\SI{200}{\hertz}$. This poses high demands on an efficient \textbf{trigger system} which is organised in three levels. Level $1$ uses only a subset of the total detector information making basic decisions to flag so called regions of interest, i.e. coordinate regions. Searches include patterns for high transverse momenta of muon tracks, electrons and photons as well as jets or large missing energy balances. The output rate after this first selection accounts for $\SI{75}{\kilo\hertz}$. The high level trigger $2$ and $3$ are responsible for selecting the level $1$ triggerd regions at full granularity and precision. The level $3$ event filter is the final stage and achieves data reduction down to the final data-taking rate of $\SI{200}{\hertz}$, writing events of the size of approximately $\SI{1.3}{\mega\byte}$ to the disks. The event filter`s selection criteria are implemented using offline analysis procedures. \cite{ATLASJINST}%efficient triggering with sufficient background rejection ##############
%evtl doch das fettgedruckte als einzelne Unterpunkte und dann etwas ausführlicher. Allerdings wird das dann ganz schön lang...
%%%%%%%%%%%%%%%%%%%%%%%%%%%%%%%%%%%%%%%%%%%%%%%%%%%%%%%%%%%%%%%%%%%%%%%%%%%%%%%%%%%%%%%%%%%%%%%%%%%%%%%
\section{Leptoquark pair production in proton-proton collisions}\label{LQpp}
%%%%%%%%%%%%%%%%%%%%%%%%%%%%%%%%%%%%%%%%%%%%%%%%%%%%%%%%%%%%%%%%%%%%%%%%%%%%%%%%%%%%%%%%%%%%%%%%%%%%%%%%
\chapter{Turning detector signatures into physical objects}
%Vorgeplänkel
%e,l,tau, jet vor bjet natürlich + MC chapter
%%%%%%%%%%%%%%%%%%%%%%%%%%%%%%%%%%%%%%%%%%%%%%%%%%%%%%%%%%%%%%%%%%%%%%%%%%%%%%%%%%%%%%%%%%%%%%%%%%%%%%%%%%%%%%%%%%
\section{b-tagging at ALTAS -- identifying b-jets}\label{btagging}
The third generation quarks, i.e. top (t) and bottom (b), play a crucial role in the Standard Model and its various extension possibilities like the Leptoquark Model due to their large masses \cite{Hansson}. Therefore, it is essential to identify hadrons containing b quarks and seperating them from light-flavour quarks at hadron collider detectors like {\ATLAS}. This task is commonly reffered as b-tagging and can be seen as a classification problem with the goal to assign right jet flavours. To that end the particle tracks in the Inner Detector and the jet reconstruction of clusters in the electromagnetic and hadronic calorimeter are discriminating objects. \cite{Paganini}\par
%
\begin{figure}[htbp]                                 
 \begin{center}                                       
  \includegraphics[width=0.55\linewidth]{figures/btagged.pdf} 
   \caption[Tracks in a b-jet.]{Signature of a b-jet with the primary and secondary vertex created relevant for b-tagging. $d_0$ is the impact parameter. \cite{Hansson}}
  \label{btagged}                                    
 \end{center}
\end{figure}
%
The long lifetime of B hadrons in the order of $\SI{1.6}{\pico\second}$ allow them to travel a few millimeters in the detector. The subsequent decay of those heavy particles within a secondary vertex produce tracks with comparably large impact parameter $d_0$ that is the shortest distance of the particle track from the primary vertex (see figure \ref{btagged}). This signature and the deduced impact parameter significance $S(d_0)=\frac{d_o}{\sigma(d_0)}$, where $\sigma(d_0)$ is the uncertainty of the impact parameter, are used by the b-tagging algorithms including five low-level and two high-level taggers. \cite{Hansson} The b-tagging algorithms rely on multivariate combinations of the information and process them to calculate a discriminant value for each jet. Thresholds on these values are then defining the working point to provide efficiencient identification of b-jets. For better information processing of the combinations of large input parameters neural network classes are used. \cite{Luca} One example for such a trained network is the MV2 tagger which uses 24 input variables of the low-level taggers together with kinematic properties\footnote{For further details on MV2 see \cite{MV2}}. \cite{Paganini}
%Grundsatzidee
%allgemeines Kapitel hierzu: https://arxiv.org/abs/1709.01290, https://arxiv.org/abs/1111.4190 und https://arxiv.org/abs/0809.4896
% dann ganz kurzer Ausblick auf https://arxiv.org/abs/1711.08811 (sind alle noch nicht in der bib)
%%%%%%%%%%%%%%%%%%%%%%%%%%%%%%%%%%%%%%%%%%%%%%%%%%%%%%%%%%%%%%%%%%%%%%%%%%%%%%%%%%%%%%%%%%%%%%%%%%%%%%%%%%%%%%%%%%
\chapter{Data analysis}
%%%%%%%%%%%%%%%%%%%%%%%%%%%%%%%%%%%%%%%%%%%%%%%%%%%%%%%%%%%%%%%%%%%%%%%%%%%%%%%%%%%%%%%%%%%%%%%%%%%%%%%%%%%%%%%%%%%
\section{Current status in the search for scalar leptoquarks}
%%%%%%%%%%%%%%%%%%%%%%%%%%%%%%%%%%%%%%%%%%%%%%%%%%%%%%%%%%%%%%%%%%%%%%%%%%%%%%%%%%%%%%%%%%%%%%%%%%%%%%%%%%%%%%%%%
\section{Starting point and research question for the analysis}
%wichtig, eigene Arbeit herausstellen, was ist mein Beitrag, nicht dass das wischi waschi wird
\section{Used data and Monte Carlo samples}
%MC16a campaign and full data 2017
\section{Physical object selection}
\section{Event selection}
%%%%%%%%%%%%%%%%%%%%%%%%%%%%%%%%%%%%%%%%%%%%%%%%%%%%%%%%%%%%%%%%%%%%%%%%%%%%%%%%%%%%%%%%%%%%%%%%%%%%%%%%%%%%%
\chapter{Results}
%%%%%%%%%%%%%%%%%%%%%%%%%%%%%%%%%%%%%%%%%%%%%%%%%%%%%%%%%%%%%%%%%%%%%%%%%%%%%%%%%%%%%%%%%%%%%%%%%%%%%%%%%%%%%
\chapter{Outlook}
%%%%%%%%%%%%%%%%%%%%%%%%%%%%%%%%%%%%%%%%%%%%%%%%%%%%%%%%%%%%%%%%%%%%%%%%%%%%%%%%%%%%%%%%%%%%%%%%%%%%%%%%%%%%%
\listoffigures
\addcontentsline{toc}{chapter}{List of figures}%fügt das Bildverzeichnis zum Inhaltsverzeichnis
\listoftables
\addcontentsline{toc}{chapter}{List of tables}%fügt das Tabellenverzeichnis zum Inhaltsverzeichnis
%%%%%%%%%%%%%%%%%%%%%%%%%%%%%%%%%%%%%%%%%%%%%%%%%%%%%%%%%%%%%%%%%%%%%%%%%%%%%%%%%%%%%%%%%%%%%%%%%%%%%%%%%%%%%%%%%%
%\chapter*{Acknowledgement}
%An dieser Stelle möchte ich meinen herzlichen Dank an die Personen aussprechen, die für das Gelingen dieser Arbeit unentbehrlich waren:
 %\begin{itemize}
% \item Professor Thomas Trefzger für die sehr freundliche Aufnahme an den Lehrstuhl Physik und ihre Didaktik + NTW
% \item Professor Raimund Ströhmer für die konstruktiven Anregungen und hilfreichen Diskussionen, welche die Arbeit sehr vorangebracht haben 
%  \item Dr. Mahsana Haleem für die Betreuung meiner Master Arbeit, für die geduldigen Erklärungen zu jeder Zeit und die lehrreichen Diskussionen zum Thema meiner Arbeit und darüber hinaus
%  \item Verena Herget für die Einarbeitung in die Datenanalyse, für tausend Antworten zu Fragen aller Art, ob phyiskalischer, programmiertechnischer oder alltäglicher Natur, für bereitwilliges und kompetentes Erklären sogar per Fernanalyse, für das Korrekturlesen der Arbeit, für die äußerst nette Atmosphäre im Büro zu jeder Zeit und den Aktionen freizeittechnischer Natur. Ohne Dich wäre ich niemals soweit gekommen. Danke! 
%  \item Deb Sankar Bhattacharya
%  \item Thorben Swirski für das Korrekturlesen der Arbeit, fröhliches debuggen, viele Anregungen nebenbei und für ausführliche, fachliche Erläuterungen
%  \item Robin Boshuis, die als sehr nette Bürokollegen die Motivation hochgehalten haben, oder zur richtigen Zeit auf eine Pause bestanden haben. Man muss ja mal Lego bauen oder Wein trinken ;)
%  \item Susan Fried und Florian Treisch für die gemeinsamen Stunden, die ich zu jeder Zeit genossen habe
%  \item Denise Böhm NTW
%  \item Stephan Lück Salat, Serien, Photo
%  \item Frank Finkenberg Weinfeste
%  \item allen anderen Mitgliedern des Lehrstuhl, die stets eine Wohlfühlatmosphäre geschaffen haben und bei kleineren Problemen hilfsbereit zur Seite standen. Danke auch für die lustigen Runden beim Mittagessen =)
%  \item meiner engeren Familie allen voran meinen Eltern und Geschwistern, die stets hinter mir stehen und mich im gesamten Studium und in der Zeit dieser Arbeit unterstützt haben
%\end{itemize}
%%%%%%%%%%%%%%%%%%%%%%%%%%%%%%%%%%%%%%%%%%%%%%%%%%%%%%%%%%%%%%%%%%%%%%%%%%%%%%%%%%%%%%%%%%%%%%%%%%%%%%%%%%%%%%
%Literaturverzeichnis-----------------------------------------------------------------------------------------
\bibliographystyle{unsrt}
\bibliography{Literatur}
\addcontentsline{toc}{chapter}{Bibliography}%fügt das Literaturverzeichnis zum Inhaltsverzeichnis
\end{document}
